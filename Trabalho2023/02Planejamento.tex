% Prof. Dr. Ausberto S. Castro Vera
% UENF - CCT - LCMAT - Curso de Ci\^{e}ncia da Computa\c{c}\~{a}o
% Campos, RJ,  2023
% Disciplina: An\'{a}lise e Projeto de Sistemas
% Aluno: Rômulo Souza Fernandes

\chapterimage{planejamento.png} % Table of contents heading image
\chapter{Etapa de Planejamento}


Neste capítulo é apresentado o ciclo de vida do desenvolvimento de sistemas, onde as bases do projeto são estabelecidas. Nesta fase, são definidos os objetivos, requisitos e direcionamentos gerais para a criação do Sistema de Gerenciamento para Concessionárias de Motos. O processo de planejamento abrange diversos aspectos cruciais que garantem o sucesso do projeto como um todo.


\section{Solicita\c{c}\~{a}o do Sistema}
%%%%%%%%%%%%%%%%%%%%%%%%%%%%%%%
A solicitação do sistema desempenha um papel central no processo de planejamento do Sistema de Gerenciamento de Concessionárias de Motos. Nesta etapa fundamental, busca-se adquirir informações detalhadas sobre as necessidades, expectativas e desafios específicos que a concessionária enfrenta. O objetivo é criar um entendimento sólido das operações atuais e identificar oportunidades para aprimoramentos, bem como estabelecer os objetivos de negócios que o sistema deve atender. A solicitação do sistema é um alicerce crítico que orienta a definição dos requisitos e a identificação dos benefícios esperados.

Nesse contexto, foram identificados desafios significativos que a concessionária enfrenta:

\begin{itemize}
\item \textbf{Gestão de Estoque Desafiadora:}A concessionária lida com a complexidade de administrar um amplo inventário de motos. Esse desafio resulta em dificuldades no rastreamento da disponibilidade das motos e potencial desperdício de recursos valiosos.

\item \textbf{Processo de Vendas Manual:}O processo de vendas atualmente realizado de forma manual está resultando em atrasos nas transações, falta de clareza e inconsistências nos registros. Essas lacunas estão prejudicando diretamente a satisfação dos clientes, afetando negativamente a eficiência e eficácia das vendas.

\item \textbf{Comunicação Interna Fragmentada:} A ausência de um sistema centralizado está impactando a comunicação entre os diferentes departamentos da concessionária. Essa falta de integração leva a informações desatualizadas e coordenação inadequada entre as equipes, comprometendo a tomada de decisões eficazes.\\
\end{itemize}

A transição da identificação desses desafios para a definição de objetivos concretos é essencial para moldar o "Sistema de Gerenciamento de Concessionárias de Motos". Esses objetivos centrais refletem metas específicas que impulsionam o desenvolvimento desse sistema inovador, com o intuito de enfrentar diretamente os desafios mencionados e promover melhorias substanciais:
\begin{itemize}
\item\textbf{Gerenciamento Eficiente de Estoque:} O sistema busca oferecer uma visão abrangente e em tempo real do estoque de motos da concessionária. Isso visa otimizar o acompanhamento, reposição e minimizar o excesso de inventário, contribuindo para operações mais eficazes.

\item\textbf{Automatização do Processo de Vendas:}A automatização abrange todas as etapas do processo de vendas, desde cotações até documentações e pagamentos. Essa abordagem visa acelerar as transações, reduzir erros e aprimorar a experiência do cliente.

\item\textbf{Centralização de Dados de Clientes:} O sistema é projetado para manter registros detalhados dos clientes, incluindo histórico de compras, preferências e informações de contato. Isso permite um atendimento personalizado e constrói relacionamentos mais profundos com os clientes.\\
\end{itemize}


\textbf{Expectativas:}
\begin{itemize}
	\item A equipe da concessionária espera que o sistema simplifique a gestão do estoque, reduzindo o tempo gasto na busca por motos e melhorando a capacidade de atender às demandas dos clientes.
	
	\item A equipe de vendas antecipa um processo de vendas mais ágil e preciso, resultando em maior satisfação dos clientes e potencial aumento nas vendas.
	
	\item A gerência da concessionária acredita que o sistema contribuirá para aprimorar a imagem da empresa, fortalecendo sua competitividade no mercado. Esses objetivos orientam o desenvolvimento do sistema, garantindo que ele atenda às necessidades da concessionária e impulsione seus objetivos de negócios.
\end{itemize}


\section{Custos: Desenvolvimento e Operacional} 
%%%%%%%%%%%%%%%%%%%%%%%%%%%%%%%
A análise de custos desempenha um papel fundamental na avaliação da viabilidade do Sistema. Para garantir que o investimento em tecnologia seja eficaz e gere retorno, é essencial considerar tanto os custos associados ao desenvolvimento inicial quanto os custos operacionais contínuos ao longo do tempo.

\textbf{Custos de Desenvolvimento:}

Os custos de desenvolvimento representam o investimento inicial necessário para criar e implementar o sistema. Esses custos abrangem diversos aspectos essenciais para garantir que o sistema seja construído de maneira sólida e eficaz:

\begin{itemize}
	\item Desenvolvimento de Software: Investir em uma equipe competente de desenvolvedores, programadores, arquitetos de software e analistas de sistemas é crucial para a construção do sistema. Esses profissionais serão responsáveis por transformar os requisitos em código funcional, garantindo a usabilidade e a eficiência do sistema.
	
	\item Aquisição de Hardware e Software: A infraestrutura tecnológica é a base do sistema. Isso inclui a compra de servidores, computadores, dispositivos móveis e as ferramentas de software necessárias para suportar a operação do sistema. Escolher as soluções tecnológicas corretas é fundamental para garantir a estabilidade e o desempenho do sistema.
	
	\item Integração e Testes:  Uma vez desenvolvido, o sistema deve ser integrado à infraestrutura existente da concessionária. Isso requer recursos dedicados para garantir que o sistema funcione harmoniosamente com os sistemas e processos já em vigor. Além disso, testes abrangentes são realizados para identificar e corrigir quaisquer falhas ou erros antes do lançamento.
	
	\item Treinamento da Equipe: Investir em treinamento é crucial para que a equipe da concessionária se familiarize com as funcionalidades do sistema e saiba como utilizá-lo de maneira eficaz. Isso inclui treinamento técnico para os funcionários operarem o sistema de maneira adequada e treinamento operacional para maximizar o uso das suas capacidades.\\
\end{itemize}

\textbf{Custos Operacionais:}

Além dos custos iniciais de desenvolvimento, é importante considerar os custos operacionais que surgem ao longo do ciclo de vida do sistema. Esses custos estão associados à manutenção, suporte contínuo e operação diária do sistema:

\begin{itemize}
	\item Manutenção e Suporte: Manter o sistema atualizado e funcional requer custos contínuos. Isso inclui a correção de eventuais bugs e problemas de funcionamento, além de garantir que o sistema esteja alinhado com as mudanças tecnológicas e as necessidades em constante evolução da concessionária.
	
	\item Treinamento Contínuo: À medida que novos funcionários são contratados ou as funcionalidades do sistema são atualizadas, é essencial fornecer treinamento contínuo para a equipe. Isso garante que todos os membros da equipe possam usar o sistema eficazmente e aproveitar todos os recursos disponíveis.
	
	\item Infraestrutura Tecnológica: A manutenção dos servidores, atualizações de software, garantia de segurança cibernética e gerenciamento de banco de dados são elementos críticos dos custos operacionais. Uma infraestrutura sólida e segura é fundamental para a continuidade das operações. 
	
	\item Licenças de Software: Caso o sistema utilize software de terceiros que exija licenciamento, esses custos também devem ser considerados. As licenças de software garantem o uso legal e a disponibilidade contínua das ferramentas essenciais para o sistema.
\end{itemize}


\section{Benef\'{\i}cios}
%%%%%%%%%%%%%%%%%%%%%%%%%%%%%%%
O Sistema de Gerenciamento para Concessionárias de Motos traz consigo uma série de benefícios específicos que têm o potencial de transformar a maneira como a concessionária opera e se relaciona com seus clientes. Esses benefícios podem ser divididos em duas categorias distintas: tangíveis e intangíveis.

       \subsection{Benef\'{\i}cios Tang\'{\i}veis}
		\begin{itemize}
			\item \textbf{Aumento nas Vendas e Lucratividade:} Ao agilizar o processo de vendas e acompanhamento, o sistema oferece à concessionária a capacidade de atender os clientes de maneira mais eficiente e eficaz. Esse aprimoramento na experiência do cliente pode resultar em um aumento substancial nas vendas e, consequentemente, na lucratividade da concessionária.
			
			\item \textbf{Otimização de Estoques:} Um dos desafios enfrentados pelas concessionárias é a gestão de estoques. Com o sistema, o controle preciso do inventário é possível, o que leva à redução dos custos associados a excessos ou falta de motos. Isso resulta na otimização da gestão de ativos e capital, contribuindo para uma operação mais eficiente.
			
			\item \textbf{Redução de Custos Operacionais:} A automação de processos internos proporcionada pelo sistema tem um impacto direto na redução dos custos operacionais. A eliminação de tarefas manuais demoradas e suscetíveis a erros libera recursos internos, economiza tempo e reduz os custos associados à mão de obra e recursos utilizados.
			
		\end{itemize}

       \subsection{Benef\'{\i}cios Intang\'{\i}veis}
		\begin{itemize}
			\item \textbf{Melhoria da Organização:} O sistema oferece uma visão abrangente das operações da concessionária. Isso melhora a organização interna ao fornecer uma representação clara dos processos e fluxos de trabalho. A equipe ganha uma compreensão mais profunda das operações, o que facilita a tomada de decisões informadas e estratégicas.
			
			\item \textbf{Atendimento de Qualidade:}A capacidade de oferecer um atendimento ágil e personalizado é ampliada com o sistema. Isso fortalece o relacionamento com os clientes, criando uma experiência positiva e satisfatória. Clientes bem atendidos têm maior probabilidade de se tornarem fiéis e recomendar a concessionária a outros.
			
			\item \textbf{Eficiência e Produtividade:}  A automação de processos não apenas reduz os custos operacionais, mas também aumenta a eficiência e produtividade da equipe. Ao automatizar tarefas rotineiras e demoradas, os funcionários podem se concentrar em atividades de maior valor agregado, impulsionando a eficiência geral da concessionária.
			
			\item \textbf{Imagem Positiva:} A modernização das operações por meio do uso do sistema pode ter um efeito direto na imagem da concessionária. A adoção de tecnologia para aprimorar os serviços e processos transmite uma imagem de inovação e confiança aos clientes. Isso pode influenciar positivamente a percepção da concessionária e sua posição no mercado.\\
		\end{itemize}
	
Em conjunto, esses benefícios tangíveis e intangíveis contribuem para transformar a concessionária, aumentando sua competitividade, eficiência e satisfação do cliente. A avaliação desses benefícios, juntamente com os custos associados ao sistema, é fundamental para determinar o impacto geral do sistema e sua viabilidade no contexto da concessionária.

\section{An\'{a}lise de Custos e Benef\'{\i}cios}
%%%%%%%%%%%%%%%%%%%%%%%%%%%%%%%
A análise de custos e benefícios desempenha um papel fundamental na avaliação abrangente da viabilidade e potencial retorno do investimento no Sistema de Gerenciamento de Concessionárias de Motos. Essa etapa crítica envolve uma avaliação detalhada dos custos associados ao desenvolvimento, implementação e operação contínua do sistema, bem como dos benefícios esperados ao longo do tempo.

\textbf{Custos do Projeto:}

A análise de custos abrange diversos aspectos que constituem o investimento necessário para trazer o sistema à vida. Entre os principais componentes de custos estão:

\begin{itemize}
	\item Desenvolvimento de Software: Alocar recursos financeiros para a equipe de desenvolvimento, programadores, arquitetos de software e analistas de sistemas responsáveis pela criação do sistema.
	
	\item Aquisição de Hardware e Software: Investir em servidores, computadores, dispositivos móveis e ferramentas de software essenciais para a infraestrutura do sistema.
	
	\item Integração e Testes: Recursos dedicados para integrar o sistema com a infraestrutura existente e realizar testes rigorosos para garantir a funcionalidade e confiabilidade.
	
	\item Treinamento da Equipe: Investir em treinamentos para capacitar a equipe da concessionária a utilizar o novo sistema de maneira eficiente.\\
	
\end{itemize}

\textbf{Custos Operacionais:}

Além dos custos iniciais de desenvolvimento, é vital considerar os custos operacionais contínuos que surgirão após a implementação do sistema. Estes incluem:

\begin{itemize}
	\item Manutenção e Suporte: Alocação de recursos para manter o sistema atualizado, corrigir erros, oferecer suporte técnico e garantir a segurança dos dados.
	
	\item Treinamento Contínuo: Recursos destinados a treinar novos membros da equipe e manter a equipe existente atualizada sobre as funcionalidades do sistema.
	
	\item Infraestrutura Tecnológica: Despesas relacionadas à manutenção dos servidores, atualizações de software, segurança cibernética e gerenciamento de banco de dados.
	
	\item Licenças de Software: Custos associados às licenças de software utilizadas no sistema.\\
\end{itemize}

\textbf{Benefícios Esperados x Análise de Custos:}

A análise de custos e benefícios busca equilibrar os custos associados ao projeto com os benefícios esperados ao longo do tempo. Isso inclui avaliar como os benefícios tangíveis e intangíveis influenciarão o retorno do investimento. Os benefícios tangíveis, como aumento nas vendas, redução de custos operacionais e otimização do estoque, podem ser quantificados em termos financeiros. Os benefícios intangíveis, como melhoria da organização, atendimento de qualidade e eficiência, têm um impacto substancial, embora não sejam facilmente mensuráveis em termos monetários.

A análise de custos e benefícios é essencial para tomar decisões informadas sobre a continuidade do projeto. Ela permite determinar se os benefícios projetados superam os custos associados ao sistema, garantindo que a concessionária faça investimentos estratégicos alinhados com seus objetivos de negócios.

Vale ressaltar que uma avaliação completa de custos e benefícios é dinâmica e deve considerar projeções de longo prazo, considerando o valor que o Sistema de Gerenciamento de Concessionárias de Motos trará para a operação e competitividade da concessionária.

\section{Estudo de Viabilidade}
%%%%%%%%%%%%%%%%%%%%%%%%%%%%%%%


       \subsection{Calend\'{a}rio }

       \subsection{Cronograma }

       \subsection{Alternativas Tecnol\'{o}gicas }
        Hardware, Software, Treinamento, etc...

       \subsection{Or\c{c}amento }
       Considere as Alternativas Tecnol\'{o}gicas para fazer pelo menos 3 or\c{c}amentos diferentes



       \subsection{Resumo e Recomenda\c{c}\~{o}es}

       Considerando .............. o sistema a ser desenvolvido SIM/N\~{A}O \'{e} vi\'{a}vel do ponto de vista ...................
