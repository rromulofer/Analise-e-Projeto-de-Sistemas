% Prof. Dr. Ausberto S. Castro Vera
% UENF - CCT - LCMAT - Curso de Ci\^{e}ncia da Computa\c{c}\~{a}o
% Campos, RJ,  2023
% Disciplina: An\'{a}lise e Projeto de Sistemas
% Aluno: 
 

\chapterimage{conclusoes.png} % Table of contents heading image
\chapter{Considera\c{c}\~{o}es Finais} 

O percurso deste trabalho, desde a introdução até a fase de considerações finais, proporcionou uma imersão abrangente no desenvolvimento de um Sistema de Gerenciamento de Concessionárias de Motos. Vamos recapitular os principais pontos discutidos em cada seção:

\section{Problemas Enfrentados e Desafios Superados}

Durante a análise e planejamento do sistema, enfrentamos desafios significativos, como a definição precisa de requisitos e a análise de custos e benefícios. Superamos esses obstáculos por meio de abordagens detalhadas, revisões iterativas e consultas às melhores práticas.

\section{Resumo do Trabalho Desenvolvido}

O trabalho iniciou-se com a descrição do sistema, destacando benefícios, objetivos e uma visão geral. Identificamos componentes essenciais, como Hardware, Software, Pessoas, Banco de Dados, Documentos, Metodologias, Mobilidade e Nuvem. A etapa de planejamento abordou solicitação do sistema, custos, benefícios, análise de custos e benefícios, e estudo de viabilidade.

A etapa de análise explorou requisitos do sistema, envolvimento de stakeholders, entrevistas, casos de uso e modelagem do sistema. O projeto do sistema focou em estratégia, objetivos, metodologia de desenvolvimento, cronograma, orçamento, equipe e refinamento dos diagramas DFD e E-R.

Na seção de arquitetura do sistema, elaboramos diagramas detalhados, incluindo Arquitetura do Sistema, Arquitetura do Hardware e Arquitetura de Software, fornecendo uma compreensão clara dos componentes e suas interações.

\section{Aspectos Não Considerados e Possíveis Estudos Futuros}

Apesar da abrangência do trabalho, reconhecemos que há áreas não exploradas, como aprofundamento na segurança do sistema, integração com tecnologias emergentes e foco na usabilidade e experiência do usuário. Estes podem ser alvos de estudos futuros para aprimorar ainda mais o sistema.

\section{Reflexão sobre o Trabalho}

Este trabalho não apenas ofereceu uma visão prática de engenharia de software, mas também enfatizou a importância da colaboração e comunicação eficaz. A integração de diversas disciplinas, desde a identificação de requisitos até a arquitetura, ilustra a complexidade envolvida no desenvolvimento de sistemas de grande escala.

\section{Conclusão}

Em conclusão, a jornada de construção do Sistema de Gerenciamento de Concessionária de Motos proporcionou aprendizados valiosos e insights que podem ser aplicados em projetos futuros. O alinhamento com as melhores práticas e a adaptabilidade às necessidades do sistema foram cruciais para o sucesso deste trabalho.
