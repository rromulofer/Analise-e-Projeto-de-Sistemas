% Prof. Dr. Ausberto S. Castro Vera
% UENF - CCT - LCMAT - Curso de Ci\^{e}ncia da Computa\c{c}\~{a}o
% Campos, RJ,  2023
% Disciplina: An\'{a}lise e Projeto de Sistemas
% Aluno:


\chapterimage{sistemas.png} % Table of contents heading image
\chapter{ Introdu\c{c}\~{a}o}

\textit{An\'{a}lise e Projeto de Sistemas} \'{e} uma disciplina orientada a descrever as duas primeiras etapas do Ciclo de Vida de Desenvolvimento de um Sistema (CVDS), neste caso, um sistema computacional.  As refer\^{e}ncias bibliogr\'{a}ficas b\'{a}sicas a serem consultadas s\~{a}o: \cite{Dennis2014}, \cite{Dennis2019} \cite{Gane1983} e \cite{Sommerville2011}. Como bibliografia complementar ser\~{a}o considerados: \cite{Satzinger2012}, \cite{Shelly2012}, \cite{Valacich2020}, \cite{Kendall2020}, \cite{Budgen2021} e \cite{Engholm2013}.

Neste documento apresentamos, passo a passo,  as atividades relacionadas com a An\'{a}lise e Design do sistema....


 \section{Descri\c{c}\~{a}o do Sistema Computacional a desenvolver}

        \subsection{abcde} 


        \subsection{defgh}

 \section{Identificando as componentes do meu sistema}

      Nesta se\c{c}\~{a}o ser\~{a}o incluidos lista de componentes (texto) bem como ilustra\c{c}\~{o}es de cada um (figuras ou imagens adequadas)
     \subsection{Componente: Hardware}


     \subsection{Componente: Software}

     \subsection{Componente: Pessoas}

     \subsection{Componente: Banco de Dados}

     \subsection{Componente: Documentos }

     \subsection{Componente: Metodologias ou Procedimentos}

     \subsection{Componente: Mobilidade}

     \subsection{Componente: Nuvem}



